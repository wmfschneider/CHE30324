\message{ !name(Outline.tex)}\documentclass[11pt]{article}
\usepackage[left=1in, right=1in, top=1in, bottom=1in, nohead]{geometry} % see geometry.pdf on how to lay out the page. There's lots.
\geometry{letterpaper} % or letter or a5paper or ... etc
% \geometry{landscape} % rotated page geometry
\usepackage{outline}
\usepackage{amsmath}
% See the ``Article customise'' template for come common customisations
\usepackage{graphicx}
\usepackage{epstopdf}
\usepackage{fancyhdr}
\usepackage[labelfont=bf]{caption}

\setlength{\headheight}{15.2pt}

\pagestyle{fancy}
\fancyhf{}
\renewcommand{\headrulewidth}{0.5pt}
\renewcommand{\footrulewidth}{0.5pt}
\lfoot{\today}
\cfoot{\copyright\ 2013 W. F. Schneider}
\rfoot{\thepage}
\lhead{\em{Physical Chemistry for Chemical Engineers}}
\rhead{ND Chem 30324}

\title{Chem 30324 outline}
\author{Wm. F. Schneider}
%\date{} % delete this line to display the current date

%%% BEGIN DOCUMENT
\begin{document}

\message{ !name(Outline.tex) !offset(-3) }


\maketitle
%\tableofcontents
\begin{outline}
\item{{\bf Lecture 0: Introduction}}
  \begin{outline}
  \item{Burning lighter example}
  \item{Foundations of physical chemistry:}
    \begin{outline}
    \item{Quantum mechanics}
    \item{Statistical mechanics}
    \item{Thermodynamics, kinetics, spectroscopy}
    \item{Physical and chemical properties of matter}
    \end{outline}
  \end{outline}

\begin{table}
\begin{center}
\caption{Key units in Physical Chemistry}
\begin{tabular}{|lrlrl|} 
\hline
$N_\mathrm{Av}$: & $6.02214 \times 10^{23}$& mol$^{-1}$  & & \\
1 amu: & $1.6605\times 10^{-27}$ & kg & & \\
$k_\mathrm{B}$: & $1.38065\times 10^{-23}$ & J~K$^{-1}$ & $8.61734\times
10^{-5}$ & eV K$^{-1}$\\
$R$: & 8.314472 & J K$^{-1}$ mol$^{-1}$ & $8.2057 \times 10^{-2}$ & l atm mol$^{-1}$ K$^{-1}$\\
$\sigma_\mathrm{SB}$: & $5.6704\times 10^{-8}$ & J s$^{-1}$ m$^{-2}$ K$^{-4}$ & & \\
$c$: & $2.99792458\times 10^8$ & m s$^{-1}$ & & \\
$h$: & $6.62607\times 10^{-34}$ & J s & $4.13566\times 10^{-15}$ & eV s
\\
$\hbar$: & $1.05457\times 10^{-34}$ & J s & $6.58212\times 10^{-16}$&  eV s \\
$hc$: & 1239.8 & eV nm  & & \\
$e$: & $1.60218\times 10^{-19}$ &  C & & \\
$\epsilon_0$: & $8.85419 \times 10^{-12}$ & C$^2$ J$^{-1}$ m$^{-1}$ & $5.52635\times
10^{-3}$ & $e^2$ \AA$^{-1}$ eV$^{-1}$ \\
$e^2/4\pi\epsilon_0$: & $2.30708 \times 10^{-28}$&  J m & 14.39964 & eV \AA\\
$a_0$: & $0.529177 \times 10^{-10}$ & m & 0.529177 & \AA\\
$E_\mathrm{H} $: & 1 & Ha & 27.212 & eV \\
\hline
\end{tabular}
\end{center}
\end{table}


\item{{\bf Lecture 1: Basic statistics}}
  \begin{outline}
  \item{Discrete probability distributions---Coin flip}
    \begin{outline}
    \item Example of Bernoulli trial, $2^n$ possible outcomes from
      $n$ flips
    \item Number of ways to get $i$ heads in $n$ flips, $_nC_i=n!/i!(n-i)!$
    \item Probability of $i$ heads $P_i \propto\ _nC_i$
    \item {\em Normalized} probability, $\tilde P_i = P_i/\sum_i P_i =\ _nC_i/2^n$
    \item Expectation value $\langle i \rangle = \sum_i i \tilde P_i$
   \end{outline}

  \item{Continuous distributions---temperature}
    \begin{outline}
      \item Probability density $P(x)$ has units $1/x$
     \item Normalized $\tilde P(x) = P(x)/\int P(x) dx$
     \item (Unitless) probability $ a < x < b = \int_a^b \tilde P(x) dx$
     \item Expectation value $\langle f(x) \rangle = \int f(x) \tilde P(x) dx$
      \item Mean $= \langle x \rangle$
      \item Mean squared $= \langle x^2 \rangle$
      \item Variance $\sigma^2=\langle x^2 \rangle - \langle x \rangle^2$
      \item Standard deviation $\Delta x = \sigma$
 \end{outline}

 
 \begin{table}\small
\begin{center}
\caption{Energy conversions and correspondences}
\begin{tabular}{|l|ccccc|}
\hline 
 & J & eV &  Hartree & kJ mol$^{-1}$ & cm$^{-1}$\\
\hline
1 J = & 1 & $6.2415\times 10^{18}$ & $2.2937\times 10^{17}$ &  $6.0221 \times
10^{20}$  & $5.0340 \times 10^{22} $\\ 
1 eV = & $1.6022 \times 10^{-19} $ & 1 & 0.036748 & 96.485 & 8065.5 \\
1 Ha = & $4.3598\times 10^{-18}$ & 27.212 & 1 & 2625.6 & 219474.6 \\
1 kJ mol$^{-1}$ = & $1.6605\times 10^{-21}$ & 0.010364 & $ 3.8087\times 10^{-4}$ & 1 & 83.5935 \\
1 cm$^{-1}$ = &$ 1.986410^{-23}$ & $1.23984\times 10^{-4}$ & $4.55623\times
10^{-6}$& 0.011963 & 1 \\
\hline 
\end{tabular}
\end{center}
\end{table}

\begin{figure}
\begin{center}
\includegraphics[scale=1.25]{Images/Boltzmann.pdf}
\caption{Boltzmann probability distribution at various temperatures}
\end{center}
\end{figure}

\item{Boltzmann distribution}
    \begin{outline}
    \item $P(E) \propto e^{-E/k_BT}$, in some sense the definition of temperature
    \item Energy and its units
    \item Absolute temperature and its units
    \item $k_BT$ as an energy scale, $\approx 0.026$~eV at 298~K

   \item{Gravity example}
      \begin{outline}
      \item $E(h)=mgh$, linear, continuous energy spectrum
      \item{molecule vs.\ car in a gravitational field}
      \item{Barometric law for gases, $P=P_0e^{-mgh/k_BT}$}
      \end{outline}
    \item {Kinetic energy in 1-D example}
      \begin{outline}
      \item $KE = \frac{1}{2}m v_x^2$
      \item $P_{1D}(v_x) = \left ( \frac{m}{2\pi k_B T} \right )^{1/2}\exp\left
          (-\frac{m|v_x|^2}{2 k_BT} \right )$
      \item Gaussian distribution,
        $G(x)=\frac{1}{\sigma\sqrt{2\pi}} \exp\left (
          -\frac{(x-\mu)^2}{2\sigma^2} \right )$, mean $\mu$, variance $\sigma^2$
      \item By inspection, $\mu=\langle v_x \rangle=0$, $\sigma^2=\langle v_x^2\rangle =k_BT/m$
      \item Molecule vs.\ car again
      \end{outline}
    \item Equipartition -- energy freely exchanged between all degrees
      of freedom
    \end{outline}
  \end{outline}

\begin{figure}
\begin{center}
\includegraphics[scale=1.25]{Images/Gaussian.pdf}
\caption{One-dimensional (Gaussian) velocities of N$_2$ gas}
\end{center}
\end{figure}

\begin{table} 
\begin{center}
    \caption{Kinetic theory of gases key equations}
    \begin{tabular}{|lr|}
     \hline
 & \\
Boltzmann distribution & $\displaystyle P(E) = g(E) e^{-E/k_BT}$ \\ \ \ \ \ ($g(E)$: degeneracy of
$E$) & \\ 
Maxwell-Boltzmann distribution & $ \displaystyle
P_{\rm MB}(v) = 4\pi v^2 \left( \frac{m}{2\pi k_B T}\right)^{3/2}\exp\left(-\frac{m
    v^2}{2k_B T}\right) $ \\  & \\
Mean and RMS speeds & 

$\displaystyle \langle v \rangle = \left( \frac{8 k_B T}{\pi m} \right)^{1/2} \ \ \ \ \langle v^2
\rangle^{1/2} = \left( \frac{3 k_B T}{m} \right)^{1/2} $ \\  & \\

Pressure & $
\displaystyle \langle P \rangle = \frac{\Delta p}{\Delta t} = m \frac{N}{V}\frac{1}{3}\langle v^2
\rangle = \frac{N k_B T}{V}=\frac{n R T}{V} $ \\ & \\ 

Wall collision frequency &
$ \displaystyle  J_W = \frac{1}{4}\frac{N}{V}\langle v \rangle=\frac{P}{\left( 2 \pi m k_B
    T\right)^{1/2}} $ \\ & \\

Molecular collision frequency &
$ \displaystyle  z=\sqrt{2} \sigma \langle v \rangle\frac{N}{V} = \frac{16\sigma P}{\left( 2\pi m k_B T
  \right)^{1/2}} $ \\ & \\

Total collisions &
$ \displaystyle z_{AA} = \frac{1}{2} \frac{N}{V} z$ \\ & \\

Mean free path &
$\displaystyle \lambda = \frac{ \langle v \rangle}{z} = \frac{V}{\sqrt{2} \sigma N} $
\\ & \\

Graham's effusion law & $\displaystyle \frac{dN}{dt}=\text{Area}\cdot  J_w \propto 1/m^{1/2} $
\\ & \\
Effusion from a vessel & $\displaystyle P=P_0 e^{-t/\tau}, \tau = \frac{V}{A}\left
  (\frac{2\pi m}{k_B T}\right )^{1/2} $ \\ & \\ 

Self-diffusion constant &
$\displaystyle D_{11} = \frac{1}{3}\langle v \rangle \lambda $ \\ & \\

Diffusion rate &
$\displaystyle \langle x^2 \rangle^{1/2} = \sqrt{2 D t} $\ \ \ \  $\langle r^2 \rangle^{1/2} = \sqrt{6
D t}$ \\ & \\

Einstein-Smoluchowski equation & $\displaystyle D_{11}= \frac{\delta^2}{2\tau}$ \\ & \\

Stokes-Einstein equation for liquids & $\displaystyle D_{11}=\frac{k_BT}{4\pi\eta r}$\ \ \
``Slip'' boundary \\
 & \\
 & $\displaystyle D_\mathrm{Brownian}=\frac{k_BT}{6\pi\eta r}$\ \ \ ``Stick'' boundary \\
\hline
    \end{tabular}
\end{center}
 \end{table}

\item{{\bf Lecture 2: Kinetic theory of gases}}
  \begin{outline}
  \item{Postulates}
    \begin{outline}
    \item Gas is composed of molecules in constant random, thermal motion
    \item Molecules only interact by perfectly elastic collisions
    \item Volume of molecules is $<<$ total volume
    \end{outline}

\begin{figure}
\begin{center}
\includegraphics[scale=1.25]{Images/MaxwellBoltzmann.pdf}
\caption{Maxwell-Boltzmann speed distribution of N$_2$ gas}
\end{center}
\end{figure}

 \item{Maxwell-Boltzmann distribution}
    \begin{outline}
     \item{3-dimensional gas, degeneracy, and $P_{MB}$}
      \item{mean speeds $\propto \sqrt{T}$ }
      \item{mean energy $U=3/2 RT$ and heat capacity $C_v=3/2 R$}
    \end{outline}
  \item{Flux and pressure}
    \begin{outline}
    \item{Velocity flux $j(v_x) dv_x= v_x \frac{N}{V}P(v_x)dv_x$, molecules /area /time /$v_x$}
    \item{Wall collisions, $J_w$, total collisions /area /time}
    \item{Momentum exchange, pressure, ideal gas law}
    \end{outline}
  \item{Collisions and mean free path}
    \begin{outline}
   \item{Collision cross section $\sigma=\pi d^2$, size of molecule}
    \item {Molecular collisions, $z$ per molecule and $z_{\mathrm{AA}}$ per volume}
    \item{Mean free path, $\lambda$, mean distance between collisions}
    \end{outline}
  \end{outline}
 
\item{{\bf Lecture 3: Transport}}
  \begin{outline}
  \item{Effusion and Graham's law, $\text{effusion rate}\propto MW^{-1/2}$}
  \item{Fick's first law: net flux proportional to concentration gradient}
    \begin{outline}
      \item{Self-diffusion constant, $D=\lambda \langle v \rangle$}
    \end{outline}
  \item{Fick's second law: time evolution of concentration gradient}
  \item Knudsen diffusion, $D=l \langle v \rangle$
  \item{Random walk model}
    \begin{outline}
    \item{Binomial distribution}
    \item{Large $N$ and Stirling approximation}
    \item{Einstein-Smoluchowski relation}
    \end{outline}
  \item{Diffusion in liquids}
    \begin{outline}
    \item{Brownian motion}
    \item{Stokes-Einstein equation}
    \end{outline}
  \end{outline}

\begin{table} 
\begin{center}
    \caption{The new physics}
    \begin{tabular}{|lr|}
     \hline
 & \\
Stefan-Boltzmann Law & $\displaystyle  \int I(\lambda,T)d\lambda = \sigma_\mathrm{SB} T^4$
\\ & \\
Wien's Law & $\displaystyle \lambda_\mathrm{max}T=2897768$ nm K \\
 & \\
Rayleigh-Jeans eq& $\displaystyle I(\lambda,T) = \frac{8\pi}{\lambda^4} k_B T c $ \\ 
& \\
Blackbody irradiance & $\displaystyle I(\lambda, T) =
\frac{8\pi}{\lambda^5}\frac{hc^2}{e^{hc/\lambda k_B T}-1}$ \\ 
& \\
Einstein crystal & $\displaystyle C_v=3R \left(\frac{h\nu}{k_BT}\right )^2\frac{e^{h\nu/k_BT}}{\left
            ( e^{h\nu/k_BT}-1 \right )^2}$ \\
& \\
Photon energy & $\displaystyle \epsilon=h\nu $ \\
& \\
Rydberg equation & $\displaystyle \nu = R_H c\left (1/n^2
        -1/k^2 \right)$ \\
& \\
Bohr equations & $\displaystyle l_n=n \hbar$ \\
$\displaystyle n=1,2, \ldots $ & $\displaystyle r_n = n^2 \left ( \frac{4 \pi
    \epsilon_0 \hbar^2}{e^2 m_e} \right ) = n^2 a_0$ \\
 & $\displaystyle E_n =-\frac{m_e e^4}{8\epsilon_0^2
   h^2}\frac{1}{n^2}=-\frac{E_H}{2}\frac{1}{n^2}$ \\ 
 & $\displaystyle p_n =\frac{e^2}{4\pi\epsilon_0}\frac{m_e}{\hbar}\frac{1}{n} =
p_0 \frac{1}{n} $ \\
& \\
de Broglie equation & $\displaystyle \lambda=h/p $ \\
\hline
\end{tabular}
\end{center}
\end{table}

\item{{\bf Lecture 4: Duality and demise of classical physics}}
  \begin{outline}
    \item{Properties of waves}
      \begin{outline}
      \item traveling waves, $\psi(x,t)=A \sin(kx-\omega t)$, $k=2\pi/\lambda$, $\omega=2\pi\nu$
      \item standing waves, $\psi(x,t) = A \sin(kx) \cos(\omega t) $
      \item interference, diffraction
      \item energy proportional to amplitude squared
      \item  Expected energy of a classical oscillator, $\langle \epsilon \rangle _\nu = k_B T$ for all $\nu$
      \end{outline}
    \item{Blackbody radiation}
      \begin{outline}
      \item Hohlraum spectrum
      \item Stefan-Boltzmann law, total irradiance 
      \item Wien's displacement law
      \item{Rayleigh-Jeans and ultraviolet catastrophe}
      \item{Planck model}
        \begin{outline}
        \item Energy spectrum of oscillators are {\em quantized}, $\epsilon_\nu=nh\nu$ 
        \item Expected energy of a quantized oscillator, $\langle \epsilon \rangle_\nu = h\nu/\left (
          e^{h\nu/k_BT}-1 \right ) $
      \item Planck expression for blackbody radiation works!
      \end{outline}

    \end{outline}
  \item{Heat capacities of solids and gases}
    \begin{outline}
    \item Law of DuLong and Pettite, $C_v = 3R$, fails at low $T$
    \item Einstein solid
      \begin{outline}
      \item Quantized vibrational energy, $\epsilon_n=nh\nu$
      \item Heat capacity goes to zero at low $T$
      \end{outline}
  \end{outline}

  \item{Photoelectric effect}
    \begin{outline}
    \item Stopping potential and work function, $T =h\nu -W$
    \item Frequency and intensity dependence
    \item{Wave-particle duality}
    \item{Photon properties, $\epsilon = h\nu, p=h/\lambda$}
    \item Compton effect, light scattering of electrons changes $\lambda$
    \end{outline}
  
  \item Rutherford, planetary model of atom, and inconsistency with Maxwell's equations
  \item{Bohr model of H atom}
    \begin{outline}
    \item Discrete H energy spectrum and Rydberg formala
    \item Bohr model (the old quantum mechanics)
      \begin{outline}
      \item Stable electron ``orbits,'' quantized angular momentum
      \item Light emission corresponds to orbital jumps, $\nu=\Delta E/h$
      \item Bohr equations
      \item Comparison with Rydberg formula
      \item Failure for larger atoms
      \end{outline}
    \end{outline}
  \item{de Broglie relation}
    \begin{outline}
      \item{$\lambda=h/p$ {\em universally}}
      \item Relation to Bohr orbits
      \item Davison and Germer experiment, $e^-$ diffraction off Ni
    \end{outline}
  \end{outline}
  
\begin{table} 
\begin{center}
    \caption{\large{Postulates of Non-relativistic Quantum Mechanics}}
   \begin{description}
    \item[Postulate 1:] {{\bf The physical state of a system is completely described by
        its wavefunction $\Psi$.}  In general, $\Psi$ is a complex function of the spatial
      coordinates and time.  $\Psi$ is required to be:}
    \begin{outline}
      \item{Single-valued}
      \item {continuous and twice differentiable}
      \item {square-integrable ($\int \Psi^*\Psi d\tau$ is defined over all finite domains)}
      \item {For bound systems, $\Psi$ can always be normalized such that $\int \Psi^*\Psi d\tau=1$}
    \end{outline}

  \item[Postulate 2:]  To every physical observable quantity $M$ there corresponds a
    Hermitian operator $\hat{M}$.  {\bf The only observable values of $M$ are the
      eignevalues of $\hat{M}$.}
    \begin{center}
    \begin{tabular}[h]{ccc}
      \hline
{\bf Physical quantity} & {\bf Operator} & {\bf Expression} \\
\hline
Position $x,y,z$ & $\hat{x},\hat{y},\hat{z}$ & $x\cdot, y\cdot, z\cdot$ \\ \\
Linear momentum $p_x, \ldots$ & $\hat{p}_x,\ldots $ & $\displaystyle -i\hbar\frac{\partial}{\partial
  x},\ldots $\\
Angular momentum $l_x, \ldots$ & $\hat{p}_x,\ldots $ & $\displaystyle -i\hbar \left
  (y\frac{\partial}{\partial z}-z\frac{\partial}{\partial y}\right ), \ldots $ \\
Kinetic energy $T$ & $\hat{T}$ & $\displaystyle -\frac{\hbar^2}{2m}\nabla^2$ \\
Potential energy $V$ & $\hat{V}$ & $V({\bf r},t)$ \\
Total energy $E$ & $\hat{H}$ & $\displaystyle -\frac{\hbar^2}{2m}\nabla^2+V({\bf r},t)$\\ \\
\hline
    \end{tabular}
  \end{center}
    \item[Postulate 3:] {If a particular observable $M$ is measured many times on many
      identical systems is a state $\Psi$, the average resuts with be the expectation
      value of the operator $\hat{M}$:
      \begin{equation*}
        \langle M \rangle = \int \Psi^* (\hat{M}\Psi)d{\bf\tau}
      \end{equation*}}
    \item[Postulate 4:] {The energy-invariant states of a system are solutions of the equation
        \begin{eqnarray*}
          \hat{H}\Psi({\bf r},t) & = & i\hbar\frac{\partial}{\partial t}\Psi({\bf r},t) \\
          \hat{H} & = & \hat{T}+\hat{V}
        \end{eqnarray*}
      The time-independent, stationary states of the system are solutions to the equation
      \begin{equation*}
        \hat{H}\Psi({\bf r}) = E\Psi(\bf{r})
      \end{equation*}
}
    \item[Postulate 5:] (The {\bf uncertainty principle}.)  Operators that do not commute
      $(\hat{A}(\hat{B}\Psi)\neq\hat{B}(\hat{A}\Psi))$ are called {\em conjugate}.
      Conjugate observables cannot be determined simultaneously to arbitrary accuracy.
      For example, the standard deviation in the measured positions and momenta of
      particles all described by the same $\Psi$ must satisfy $\Delta x\Delta p_x \geq \hbar/2$.
    \end{description}
\end{center}
\end{table}

\item{{\bf Lecture 5: Postulates of quantum mechanics}}
  \begin{outline}
  \item{Schr\"{o}dinger equation describes wave-like properties of matter}
  \item{Born interpretation}
    \begin{outline}
      \item wavefunction is a probability amplitude
      \item wavefunction squared is probability density
    \end{outline}

  \item{Postulates}
    \begin{outline}
    \item{Wavefunction contains all information about a system}
    \item{Operators used to extract that information}
      \begin{outline}
      \item QM operators are {\em Hermitian}
      \item Have eigenvectors and real eigenvalues, $\hat{O}\psi_i=o\psi_i$
      \item Are orthogonal, $\langle \psi_i | \psi_j \rangle = \delta_{ij}$
      \item Always observe an eigenvalue when making an observation
      \end{outline}
    \item{Expectation values}
    \item{Energy-invariant wavefunctions given by Schr\"odinger equation}
    \item{Uncertainty principle}
    \end{outline}
  \item{Particle in a box illustrations}
  \end{outline}

\item{{\bf Lecture 6: Particle in a box model}}
  \begin{outline}
 \item{Particle between infinite walls, electron confined in a wire}
 \item Classical solution, either stationary or uniform bouncing back and forth
  \item{One-dimesional QM solutions}
    \begin{outline}
    \item Schr\"{o}dinder equation and boundary conditions
    \item discrete, quantized solutions
     \item standing waves, $\lambda=2 L/n$, $n-1$ nodes, non-uniform probability
      \item Ho paper, STM of Pd wire
      \item zero point energy and uncertainty
      \item correspondence principle
      \item superpositions
    \end{outline}
  \item{Finite walls and tunneling}
    \begin{outline}
    \item Potential well of finite depth $V_0$
    \item Finite number of bound states
    \item Classical region, $\psi(x) ~ e^{ikx}+e^{-ikx}, k=\sqrt{2mE}/\hbar$
    \item ``Forbidden'' region, $\psi(x) ~ e^{\kappa x}+e^{-\kappa x},
      \kappa=\sqrt{2m(V_0-E)}/\hbar$
    \item Non-zero probability to ``tunnel'' into forbidden region
    \item Tunneling between two adjacent wells: chemical bonding, STM, nanoelectronics
    \item H atom tunneling: NH$_3$ inversion, H transfer, kinetic isotope effect
    \end{outline}

  \item{Multiple dimensions}
    \begin{outline}
    \item separation of variables
    \end{outline}
  \item Introduce Pauli principle for fermions?

  \end{outline}
\begin{table}[tb]
   \begin{center}
   \caption{Particle-in-a-box model}
    \label{Particle-in-a-box}
\begin{tabular}[h]{|c|}
\hline
 \\
$\displaystyle       V(x) = \left \{
        \begin{array}{rl}
          0 & 0 < x < L \\
          \infty & x \leq 0 \text{ or } x \geq L
        \end{array} \right . $ \\
 \\
$\displaystyle     \psi_n(x) =\sqrt{\frac{2}{L}} \sin \left ( \frac{n\pi x}{L} \right )$
\\ 
 \\
$\displaystyle     E_n =\frac{n^2\pi^2\hbar^2}{2mL^2}, n = 1, 2, ...$ \\
 \\
     \includegraphics[scale=.6]{Images/PIB} \\       
\hline
\end{tabular}
 \end{center}
\end{table}


\item{{\bf Lecture 7: Harmonic oscillator}}
  \begin{outline}
  \item{Classical harmonic oscillator}
    \begin{outline}
    \item Hooke's law, $F=-k(x-x_0)$, $k$ spring constant
   \item Continuous sinusoidal motion
    \item $x(t)=A \sin(\frac{k}{\mu})^{1/2}t, \nu=\frac{1}{2\pi}(\frac{k}{\mu})^{1/2}, E=\frac{1}{2}kA^2$
    \item Exchanging kinetic and potential energies
    \end{outline}
  \item Quantum harmonic oscillator
    \begin{outline}
    \item Solutions like P-I-A-B, waves, nodes
   \item Zero-point energy
    \item Expectation values $\langle x^2 \rangle =
      \alpha^2 (v+1/2), \langle V(x) \rangle = \frac{1}{2} h\nu (v+\frac{1}{2})$
    \item Classical turning point and tunneling
    \item Classical limiting behavior
    \end{outline}
  \item HCl example
    \begin{outline}
          \item Reduced mass, $\frac{1}{\mu}=\frac{1}{m_A}+\frac{1}{m_B}$
    \end{outline}
    \item Anharmonicity, Morse potential
  \end{outline}

\begin{table}[tbh]
   \begin{center}
   \caption{Harmonic oscillator model}
    \label{Harmonic-oscillator}
\begin{tabular}[h]{|c|}
\hline
 \\
$\displaystyle       V(x) = \frac{1}{2} k x^2, -\infty < x < \infty $ \\
 \\
$\displaystyle     \psi_v(x) = N_v H_v(x/\alpha)e^{-x^2/2\alpha^2}, v = 0, 1, 2, \ldots $ \\
\\
$\displaystyle \alpha=(\hbar^2/\mu k)^{1/4}, N_v=(2^vv!\alpha\sqrt{\pi})^{-1/2} $ \\
 \\
$\displaystyle  H_v(x/\alpha)=$ Hermite polynomials \\
$\displaystyle H_0(y) =1, H_1(y) = 2y, H_2(y) = 4y^2-2, \ldots $ \\ 
 \\
$\displaystyle     E_v=(v+\frac{1}{2})h \nu, v=0, 1, 2, ...$ \\
 \\
     \includegraphics[scale=.6]{Images/HO} \\       
\hline
\end{tabular}
 \end{center}
\end{table}

\item{{\bf Lecture 8: Vibrational spectroscopy}}
  \begin{outline}
    \item Spectroscopy, measuring interaction of light with matter, $I(\nu)/I(\nu_0)$
    \item Bohr condition, $|E_f-E_i|/h=\nu =c\tilde{\nu}=c/\lambda$
    \item Intensities determined by state populations and transition probabilities
    \item Einstein coefficients
      \begin{outline}
        \item Stimulated absorption, $dn_1/dt= -n_1 B\rho(\nu)$
        \item Stimulated emission, $dn_2/dt= -n_2 B\rho(\nu)$
        \item Spontaneous emission, $dn_2/dt=-n_2 A, A=\left ( \frac{8\pi h
              \nu^3}{c^3}\right )B$
        \item $1/A=$ lifetime
      \end{outline}

    \item Transition probability
      \begin{outline}
        \item Einstein coefficient $B_{if}=\frac{|\mu_{if}|^2}{6\epsilon_0\hbar^2}$
        \item Classical electric dipole, $\overrightarrow{\mu}=q \cdot
          \overrightarrow{l}$, quantum dipole operator $\hat\mu = e\cdot \overrightarrow{r}$
        \item Transition dipole moment, $\mu_{if} = \left(
        \frac{d\mu}{dx}\right ) \langle \psi_i|\hat\mu |\psi_f \rangle $
    \item Selection rules---conditions that make $\mu_{if}$ non-zero,
      ``allowed'' vs. ``forbidden'' transitions
      \end{outline}

    \item Vibrational transitions
      \begin{outline}
        \item Gross selection rule: dynamic dipole $d\mu/dx$ non-zero
        \item Specific selection rule: dipole integral $\langle \psi_v|\hat\mu|\psi_{v^\prime} \rangle =0$
          unless $\Delta v = \pm 1$
        \item Allowed $\Delta E = h\nu$
        \item Boltzmann, $v=1$ states dominate at normal $T$
      \end{outline}

    \item Vibrational spectroscopy
      \begin{outline}
        \item Diatomics, homo- vs. heteronuclear
        \item Polyatomics, $3n-6$ ($3n-5$ for linear polyatomic) vibrational modes
        \item CO$_2$ example
      \end{outline}

    \item Raman spectroscopy
 \end{outline}

\begin{table}[tbh]
   \begin{center}
   \caption{2-D rigid rotor model}
    \label{Rigid rotor}
\begin{tabular}[h]{|c|}
\hline
 \\
$\displaystyle       V(\phi) = 0, 0 \leq \phi \leq 2\pi $ \\
 \\
$\displaystyle \hat H = -\frac{\hbar^2}{2 I} \frac{\partial^2}{\partial
  \phi^2},\ \ \ \ \ I=\mu R^2
$\\
\\
$\displaystyle     \psi_{m_l}(\phi) = \frac{1}{\sqrt{2\pi}} e^{-i m_l \phi}, m_l
= 0, \pm 1, \pm 2, \ldots $ \\
\\
$\displaystyle     E_{m_l}=\frac{m_l^2 \hbar^2}{2 I^2}$ \\
 \\
$\displaystyle L_z = m_l \hbar$ \\
\\
     \includegraphics[scale=1]{Images/2D_rotor} \\       
\hline
\end{tabular}
 \end{center}
\end{table}

\item{{\bf Lecture 9: Rigid rotor}}
  \begin{outline}
  \item Classical rigid rotor
    \begin{outline}
      \item Rotation about an axis vs.\ linear motion
      \item Moment of intertia $I=\mu r^2$
      \item Angular momentum, ${\bf l} = I {\bf \omega}= {\bf r}\times{\bf p}$, $T=|{\bf l}|^2/2 I$
      \item Angular momentum and energy continuous variable
    \end{outline}

  \item Quantum rotor in a plane
    \begin{outline}
    \item Angular momentum and kinetic energy operators in polar coordinates,
      $\hat l_z = -i\hbar \frac{d}{d\phi}$
    \item Eigenfunctions and energy spectrum
    \item No zero point energy
    \item Angular momentum eignefunctions, $l_z = m_l \hbar$
      \item Energy superpositions and localization 
    \end{outline}
  \item Quantum rotor in 3-D
    \begin{outline}
    \item Angular momentum and kinetic energy operators in spherical coordinates
    \item Spherical harmonic solutions, $Y_{lm_l}$
    \item Azimuthal QN $l=0, 1, ...$
    \item Magnetic QN $m_l = -l, -l+1, ..., l$
    \item Energy spectrum, $2 l + 1$ degeneracy
    \item Vector model - can only know total total $|L|$ and $L_z$
    \item Wavefunctions look like atomic orbitals, $l$ nodes
    \end{outline}
  \item Particle angular momentum
    \begin{outline}
    \item Fermions, mass, half-integer spin
      \begin{outline}
      \item Electron, $s=1/2, m_s=\pm 1/2$
      \end{outline}
    \item Bosons, force-carrying, integer spin
    \end{outline}
  \item{Diatomic rotational spectroscopy}
    \begin{outline}
    \item Rotational constant $B = \hbar/4\pi I c$ cm$^{-1}$, $I=\mu R^2$
    \item Gross selection rule: dipole moment non-zero
    \item Specific selection rule: $\Delta l=\pm 1$, $\Delta m_l=0, \pm1$
    \item $\Delta \tilde E_l  = 2B(l+1)$ cm$^{-1}$
    \item Rotational state populations
    \end{outline}
  \item{Polyatomic rotational spectroscopy}
  \item{Vibration-rotation spectroscopy}
    \begin{outline}
    \item Harmonic oscillator + rigid rotor
    \item Selection rules: $\Delta v = \pm 1, \Delta l=\pm 1$
    \item $R$ branch: $\Delta \tilde E  = \tilde \nu + 2B(l+1), \Delta l = 1$ 
    \item $P$ branch: $\Delta \tilde E = \tilde \nu - 2B(l), \Delta l = -1$       
    \end{outline}

  \end{outline}
\begin{table}[tbh]
   \begin{center}
   \caption{3-D rigid rotor model}
    \label{3-D Rigid rotor}
\begin{tabular}[h]{|c|}
\hline
 \\
$\displaystyle       V(\theta,\phi) = 0, 0 \leq \phi \leq 2\pi, 0 \leq \theta <
\pi$ \\
 \\
$\displaystyle     \hat L^2 = -\hbar^2 \left [
  \frac{1}{\sin^2\theta}\frac{\partial^2}{\partial \phi^2}+\frac{1}{\sin
    \theta}\frac{\partial}{\partial \theta}\left ( \sin \theta
    \frac{\partial}{\partial \theta}\right ) \right ] $ \\
\\
$\displaystyle \hat H_\text{rot} = \frac{1}{2 I} \hat L^2$ \\
\\
$\displaystyle     Y_{lm_l}(\theta,\phi)=N_l^{|m|}P_l^{|m|}(\cos(\theta))e^{im_l\phi}$ \\
\\
$\displaystyle l = 0, 1, 2, \ldots, \ \ \ \ \ \ m_l = 0,\pm 1, \ldots, \pm l$
\\
\\
$\displaystyle     E_{l}=\frac{\hbar^2}{2 I}l(l+1)$ \\
 \\
$\displaystyle |L| = \hbar \sqrt{l(l+1)}, L_z = m_l \hbar $ \\
\\
     \includegraphics[scale=0.4]{Images/3D_rotor} \\       
\hline
\end{tabular}
 \end{center}
\end{table}

\item{{\bf Lecture 10: Hydrogen atom}}
  \begin{outline}
  \item Schr\"odinger equation
    \begin{outline}
    \item Spherical coordinates and separation of variables
    \item Coulomb potential $v_\mathrm{Coulomb}(r)=-\frac{e^2}{4\pi\epsilon_0}\frac{1}{r}$
    \item Centripetal potential  $v=\hbar^2\frac{l(l+1)}{2\mu r^2}$
    \end{outline}
    \item Solutions
    \begin{outline}
  \item{$\psi(r,\theta,\phi)=R_{nl}(r)Y_{lm}(\theta,\phi)$}
  \item Principle quantum number $n=1,2,...$
    \begin{outline}
    \item $K$, $L$, $M$, $N$, ... shells
    \item $n-1$ radial nodes
    \end{outline}
  \item Azimuthal quantum number $l=0,1,...,n-1$
    \begin{outline}
    \item $s$, $p$, $d$, ... orbital sub-shells
    \item $l$ angular nodes
    \end{outline}
  \item Magnetic quantum number $m_l=-l,-l+1,...,l$
  \item Spin quantum number $m_s=\pm 1/2$
    \end{outline}
  \item Energy spectrum and populations
  \item Electronic selection rules
    \begin{outline}
    \item $\Delta l=\pm 1$, $\Delta m_s =0$, $\Delta m_l = 0,\pm 1$
    \end{outline}
  \item Wavefunctions = ``orbitals''
  \item Radial probability function $P_{nl}(r)=r^2 R_{nl}^2(r)$
    \begin{outline}
    \item $\langle r\rangle = \int r P_{nl}(r) dr = (\frac{3}{2}n^2-l(l+1))a_0$
    \end{outline}
  \end{outline}
\begin{table}[tbh]
   \begin{center}
   \caption{Hydrogen atom}
    \label{Hydrogen atom}
\begin{tabular}[h]{|c|}
\hline
 \\
$\displaystyle       V(r) = -\frac{e^2}{4\pi\epsilon_0}\frac{1}{r}, 0 < r< \infty$ \\
 \\
$\displaystyle     \hat H = -\frac{\hbar^2}{2m_e}\frac{1}{r^2}\left [
  \frac{\partial}{\partial r}r^2\frac{\partial}{\partial r} + \hat L^2 \right ] +V(r)$ \\
\\
$\displaystyle \psi(r,\theta,\phi) = R(r)Y_{l,m_l}(\theta,\phi) $ \\
\\
$\displaystyle   \left \{ -\frac{\hbar^2}{2m_e}\frac{1}{r^2}
            \frac{d}{d r} \left ( r^2 \frac{d}{dr}\right ) + \frac{\hbar^2
              l(l+1)}{2 m_e r^2}
          -\frac{e^2}{4\pi\epsilon_0}\frac{1}{r}\right \} R(r) = E R(r) $ \\
\\
$\displaystyle R_{nl}(r) = N_{nl} e^{-x/2} x^l L_{nl}(x),\ \ \  x = \frac{2 r}{n a_0} $
\\
$\displaystyle P_{nl}(r) = r^2 R_{nl}^2 $
\\
\\
$\displaystyle n = 1, 2, \ldots,\ \  l = 0, \ldots, n-1 \ \ m_l = 0,\pm 1, \ldots, \pm l$
\\
\\
$\displaystyle     E_{n}=-\frac{1}{2}\frac{\hbar^2}{m_e a_0^2}\frac{1}{n^2} =-\frac{E_H}{2}\frac{1}{n^2}$ \\
 \\
$\displaystyle |L| = \hbar \sqrt{l(l+1)}, L_z = m_l \hbar $ \\
\\
%%     \includegraphics[scale=0.4]{Images/H_atom} \\       
\hline
\end{tabular}
 \end{center}
\end{table}

\item{{\bf Lecture 11: Many-electron atoms}}
  \begin{outline}
  \item Many-electron problem, Schr\"odinger equation not exactly solvable
    \begin{outline}
      \item $e^- -e^-$ interaction terms prevent separation of variables
    \end{outline}
  \item Independent electron model basis of all solutions, describes each
    electron by its own wavefunction, or ``orbital''
  \item Qualitative solutions
    \begin{outline}
    \item $\psi_i$ look like H atom orbitals,  labeled by same quantum numbers
    \item {\em Aufbau principle}: ``Build-up'' electron configuration by adding
      electrons into H-atom-like orbitals, from bottom up
    \item {\em Pauli exclusion principle}: Every electron in atom must have a unique
      set of quantum numbers, so only two per orbital (with opposite spin)
    \item {\em Pauli exclusion principle} (formally): The wavefunction of a
      multi-particle system must be anti-symmetric to coordinate exchange if
      the particles are fermions, and symmetric to coordinate exchange if the
      particles are bosons
    \item {\em Hund's rule}: Electrons in degenerate orbitals prefer to be
      spin-aligned.  Configuration with highest {\em spin multiplicity} is the
      most preferred
    \item Rules give the familiar structure of the periodic table
    \item Electrons in different subshells experience different effective nuclear
      charge $Z_\mathrm{eff} = Z - \sigma_{nl}$
      \begin{outline}
      \item Inner (``core'') shells not shielded well at all
      \item Inner shell electrons ``shield'' outer electrons well
      \item Within a shell, $s$ shielded less than $p$ less than $d$ ...,
        causes degeneracy to break down
      \item Electrons in same subshell shield each other poorly, causing
        ionization energy to increase across the subshell
      \end{outline}
    \end{outline}
  \item {\em Variational principle}--True wavefunction energy is lower bound on
    energy of any trial wavefunction
    \begin{outline}
      \item Because true solutions form a complete set
      \item Use to optimize candidate wavefunctions (give an example?)
    \end{outline}
  \item Quantitative solutions
    \begin{outline}
    \item Schr\"odinger equation
     \begin{outline}
     \item $\hat H \Psi({\bf r}_1, {\bf r}_2,...)=E \Psi({\bf r}_1, {\bf r}_2,...)$
     \item $\hat H = \sum_i \hat h_i + \frac{e^2}{4 \pi
         \epsilon_0}\sum_i\sum_{j>i}\frac{1}{|{\bf r}_i-{\bf r}_j|}$
     \item $\hat h_i = -\frac{\hbar^2}{2m_e}\nabla^2_i-\frac{Z
         e^2}{4\pi\epsilon_0}\frac{1}{|{\bf r}_i|}$
     \end{outline}
   \item Construct candidate many-electron wavefunction $\Psi$ from one
     electron wavefunctions (mathematical details vary with exact approach)
       \begin{outline}
       \item $\Psi({\bf r}_1, {\bf r}_2,...)\approx \psi_1({\bf
           r}_1)\psi_2({\bf r}_2)...\psi_n({\bf r}_n)$
       \end{outline}
     \item Calculate expectation value of $E$ of approximate model and apply
      {\em  variational principle} to find equations that describe ``best'' (lowest
       total energy) set of $\psi_i$
       \begin{outline}
       \item $\frac{\partial E}{\partial \psi_i}=0 \ \ \ \forall i$
       \item $\hat f\psi=\left\{\hat h + \hat v_\mathrm{Coul}[\psi_i] + \hat
           v_\mathrm{ex}[\psi_i]+\hat v_\mathrm{corr}[\psi_i] \right\}\psi=\epsilon\psi$
       \item (Motivate as equation for an electron moving in a ``field'' of
         other electrons, adding an electron to a known set of $\psi_i$)
       \item $E=\sum_i \epsilon_i-\frac{1}{2}\langle \Psi |\hat v_\mathrm{Coul}[\psi_i] + \hat
           v_\mathrm{ex}[\psi_i]+\hat v_\mathrm{corr}[\psi_i]|\Psi \rangle$
       \end{outline}
     \item Electron-electron interactions
       \begin{outline}
       \item Coulomb ($\hat v_\mathrm{Coul}$): classical electrostatic
         repulsion between distinguishable electron ``clouds''
       \item Exchange ($\hat v_\mathrm{ex}$): accounts for electron
         indistinguishability (Pauli principle for fermions).  Decreases
         Coulomb repulsion because electrons of like spin intrinsically avoid
         one another
       \item Correlation ($\hat v_\mathrm{corr}$): decrease in Coulomb
         repulsion due to dynamic ability of electrons to avoid one another;
         ``fixes'' orbital approximation
       \end{outline}
     \item General form of exchange potential is expensive to calculate; general
       form of correlation potential is unknown
       \begin{outline}
     \item {\em Hartree model}: Include only classical Coulomb repulsion $\hat
       v_\mathrm{Coul}$
     \item {\em Hartree-Fock model}: Include Coulomb and exchange
     \item {\em Density-functional theory} (DFT): Include Coulomb and
       approximate expressions for exchange and correlation         
       \end{outline}
     \item All the potential terms $\hat v$ depend on the solutions, so equations
       must be solved {\em iteratively} to {\em self-consistency}
     \end{outline}
   \item Herman-Skillman code for DFT calculations on atoms
   \end{outline}

 \item {\bf Lecture 12: Molecular orbital theory of molecules}
   \begin{outline}
   \item Clamped nucleus (``Born-Oppenheimer'') approximation
     \begin{outline}
     \item Write one-electron equations parametrically in terms of positions of
     all atoms
   \item   $\hat h_i = -\frac{\hbar^2}{2m_e}\nabla^2_i-\sum_\alpha \frac{Z_\alpha
         e^2}{4\pi\epsilon_0}\frac{1}{|{\bf r}_i-{\bf R}_\alpha|}$
     \item Solve as for atoms, using some model for electron-electron interactions
     \item Potential energy surface (PES)
       \begin{outline}
       \item $E({\bf R}_\alpha, {\bf
           R}_\beta,...)=E_\mathrm{elec}+\frac{e^2}{4\pi\epsilon_0}\sum_\alpha\sum_{\beta>\alpha}\frac{Z_\alpha
           Z_\beta}{|{\bf R}_\alpha-{\bf R}_\beta|}$ 
       \end{outline}
     \end{outline}
   \item H$_2$ molecule as perturbation on two H atoms brought from infinite distance
     \begin{outline}
       \item ``Bonding'' orbital, $\sigma_g({\bf r}) = 1{\rm s_A}+1{\rm s_B}$
       \item ``Anti-bonding'' orbital, $\sigma_u({\bf r}) = 1{\rm s_A}-1{\rm s_B}$
       \item Interaction scales with ``overlap'' $\langle 1{\rm s_A} | 1{\rm
           s_B} \rangle$
       \item Ground ``configuration'' $=\sigma_g^2$
       \item Bond order = $\frac{1}{2}(n-n^*)$
     \end{outline}
   \item Secular equations
     \begin{outline}
     \item Expand molecular orbitals in ``basis'' of atomic-like orbitals
       \begin{equation}
         \psi_\mathrm{MO}=\sum_a c_a\phi_a({\bf r})
       \end{equation}
     \item Problem reduces to finding set of $c_a$ that give best molecular
       orbitals (MOs)
     \item Substituting into Fock equation and integrating yields set of linear
       equations for the $c_a$ for each MO
       \begin{displaymath}
         \left ( \begin{array}{ccc}
           F_{11}-\epsilon S_{11} & F_{12}-\epsilon S_{12} & \ldots \\
           F_{21}-\epsilon S_{21} & F_{22}-\epsilon S_{22} & \ldots \\
           \vdots & \vdots & \vdots
         \end{array} \right ) \left (
         \begin{array}{c}
           c_1 \\
           c_2 \\
           \vdots
         \end{array} \right ) = 0
     \end{displaymath}
     \begin{outline}
     \item $F_{ij} = F_{ji} = \langle \phi_i | \hat f | \phi_j \rangle$ are Fock
       ``matrix elements''
     \item $S_{ij} = S_{ji} = \langle \phi_i | \phi_j \rangle$ are overlaps
     \item Typically basis functions normalized such that $S_{ii} = 1$
     \item $\epsilon$ are molecular orbital energies (to be solved for, as many
       as there are equations)
     \end{outline}
   \item From linear algebra, only possible solutions are those that make the
     determinant vanish
       \begin{displaymath}
         \left | \begin{array}{ccc}
           F_{11}-\epsilon S_{11} & F_{12}-\epsilon S_{12} & \ldots \\
           F_{21}-\epsilon S_{21} & F_{22}-\epsilon S_{22} & \ldots \\
           \vdots & \vdots & \vdots
         \end{array} \right | = 0
     \end{displaymath}
   \item Solve for $\epsilon$s and back-substitute to find correspond $c_i$s
   \end{outline}
 \item Qualitative solutions of secular equations
   \begin{outline}
   \item Lot's of insight into chemical bonding can be obtained from
     approximate solutions to secular equations, basis of ``molecular orbital theory''
   \item Two general assumptions
     \begin{outline}
     \item Diagonal Fock elements are approximately equal to energies of
       corresponding atomic orbitals: $F_{ii} \approx \epsilon_{i,\mathrm{ao}}$
     \item Off-diagonal elements proportional to overlap and inversely
       proportional to energy difference:
       \begin{displaymath}
         F_{ij} \propto \frac{S_{ij}}{\epsilon_{i,\mathrm{ao}}-\epsilon_{j,\mathrm{ao}}}
       \end{displaymath}
     \item (Often) set differential overlap $S_{ij}=0$
     \end{outline}
   \end{outline}
 \item H$_2$ example, again
   \begin{outline}
   \item Assign one 1s atomic orbital (``basis function'') to each atom
     \begin{eqnarray*}
       F_{11}=F_{22}=\epsilon_{1\mathrm{s}}=\alpha \\
       F_{12}=F_{21}=\beta \\
       \alpha < \beta < 0\ \ \mathrm{typically}
     \end{eqnarray*}
   \item Set-up and solve secular matrix
     \begin{displaymath}
      \left | \begin{array}{cc}
          \alpha-\epsilon & \beta-\epsilon S \\
          \beta - \epsilon S & \alpha-\epsilon
          \end{array} \right | = 0
     \end{displaymath}
     \begin{eqnarray*}
       \epsilon_+=\frac{\alpha+\beta}{1+S}, c_1=c_2=\frac{1}{\sqrt{2(1+S)}} \\
       \epsilon_-=\frac{\alpha-\beta}{1+S}, c_1=-c_2=\frac{1}{\sqrt{2(1-S)}} \\
     \end{eqnarray*}
     \begin{center}
     \includegraphics[scale=0.25]{Images/H2-MO}       
     \end{center}
   \end{outline}
 \item Heteronuclear diatomic: LiH, HF, BH example
   \begin{outline}
   \item Only AOs of appropriate symmetry, overlap, and energy match can
     combine to form MOs
   \item LiH: H 1s + Li 2s, bond polarized towards H
   \item HF: H 1s + F 2p, bond polarized towards F, lots of non-bonding orbitals
   \item BH: H 1s, B 2s and 2p$_z \rightarrow$ bonding, non-bonding, anti-bonding orbitals
   \end{outline}
 \item Homonuclear diatomic: O$_2$
   \begin{outline}
   \item Assign aos, 1s, 2s, 2p for each atom (10 total)
   \item In principle, solve $10\times 10$ secular matrix
   \item In practice, matrix elements rules mean only a few off-diagonal
     elements survive
     \begin{outline}
     \item 1s + 1s do nothing
     \item 2s + 2s form $\sigma$ bond and anti-bond
     \item 2p$_z$ + 2p$_z$ form second bond and anti-bond
     \item 2p$_{x,y}$ + 2p$_{x,y}$ form degenerate $\pi$ bonds and anti-bonds
     \item O$_2$ is a triplet, consistent with experiment!
     \end{outline}
   \end{outline}
 \item The H\"uckel/tight binding model
    \begin{outline}
   \item $F_{ii}=\alpha, S_{ij}=\delta_{ij}, F_{ij}=\beta$ iff $i$ adjacent
     to $j$
   \item Ethylene example
   \item Butadiene example
   \item Benzene example
   \item Infinite chain example      
    \end{outline}
  \item Band structure of solids
  \end{outline}

\item {\bf Lecture 13: Computational chemistry}
  \begin{outline}
  \item Numerical solvers of Schr\"odinger equation for molecules readily
    available today
  \item Have to specify:
    \begin{outline}
    \item Identity of atoms
    \item Positions of atoms (distances, angles, $\ldots$)
    \item (spin multiplicity)
    \item exact theoretical model (how are Coulomb, exchange, and correlation described?)
      \begin{outline}
      \item Hartree, Hartree-Fock, DFT (various flavors), $\ldots$
      \end{outline}
    \item basis set to express wavefunctions in terms of
    \item initial guess of wavefunction coefficients (often guessed for you)
    \end{outline}
  \item Secular equations solved iteratively until input coefficients = output coefficients
    \begin{outline}
      \item ``self-consistent field''
    \end{outline}
  \item Output
    \begin{outline}
      \item energies of molecular orbitals
      \item occupancies of molecular orbitals
      \item coefficients describing molecular orbitals
      \item total electron wavefunction, total electron density, dipole moment, $\ldots$
      \item total molecular energy
      \item derivatives (``gradients'') of total energy w.r.t. atom positions
    \end{outline}
  \item Plot total energy vs.\ internal coordinates: potential energy surface (PES)
  \item Search iteratively for minimum point on PES (by hand or using
    gradient-driven search): equilibrium geometry
  \item Find second derivative of energy at minimum point on PES: harmonic
    vibrational frequency
  \item Find energy at minimum relative to atoms (or other molecules): reaction energy
  \item H$_2$ example
    \begin{center}
      \includegraphics[scale=0.6]{Images/H2-PES}
    \end{center}
  \item Polyatomic molecules
    \begin{outline}
    \item Gradient-driven optimizations, $3n-6$ degrees of freedom
    \item Hessian matrix for frequencies
    \end{outline}
  \end{outline}

\item {\bf Lecture 14: Electronic spectroscopy}
  \begin{outline}
  \item Electronic spectroscopy examines electron jumps between energy states (“orbitals”)
  \item The orbital structure of each substance is unique, so unlike vibrational or rotational spectroscopy, there is no simple general energy model for electronic transitions.  There are a couple general rules, though:
    \begin{outline}
      \item Spin selection rule: $\Delta S = 0$
        \begin{outline}
          \item electron spins are ``forbidden to change''
        \end{outline}
      \item Koopman’s ``theorem'':
        \begin{outline}
        \item the energy of an electronic transition is approximately the difference in energy between the orbital an electron starts in and the one it ends up in
          \begin{outline}
          \item $h \nu \approx \epsilon_\mathrm{final}-\epsilon_\mathrm{initial}$
          \end{outline}
        \item this “theorem” is an approximation because the orbitals are not static; more correctly, the energy difference is given by a full electronic structure calculation on the initial and final states
        \end{outline}
      \end{outline}
    \item Various classes of transitions
      \begin{outline}
      \item UV/visible spectroscopy
        \begin{outline}
        \item electron jumps from valence filled to empty orbital
        \item energies of an eV or so
        \item $\pi$ to $\pi^*$ classic example
        \end{outline}
      \item UV photoelectron spectroscopy
        \begin{outline}
        \item electron ionized from valence filled orbital          
        \end{outline}
      \item X-ray spectroscopy
        \begin{outline}
        \item electron ionized from core orbital or promoted from core to an empty orbital 
        \item 10’s-1000’s eV energies
        \item many types, from lab scale to massive synchrotrons
        \item information about elemental composition, oxidation state, coordination, ...          
        \end{outline}
      \end{outline}

    \item Various classes of electron events
      \begin{outline}
      \item stimulated absorption
        \begin{outline}
        \item photon causes jump from lower to higher energy electronic state
        \item often convoluted with jumps to different vibrational, rotational states          
        \end{outline}
      \item spontaneous emission
        \begin{outline}
        \item electron spontaneously jumps to a lower energy state and emits a photon
        \item basis of fluorescence ($\Delta S = 0$)
        \item basis of long-lived phosphorescence ($\Delta S \neq  0$)
        \item long-lived because it breaks the spin selection rule
        \end{outline}
      \item stimulated emission
        \begin{outline}
        \item passing photon causes electron to jump from higher to a lower energy state and to emit another photon
        \item cascade of such stimulated events is the basis of laser action
        \end{outline}
      \end{outline}
    \end{outline}

\item {\bf Lecture XX: Electronic and magnetic properties} - skipped

\item {\bf Lecture 15: Statistical mechanics}
  \begin{outline}
    \item Need machinary to average QM information over macroscopic systems
    \item Equal {\em a priori} probabilities
    \item Two-state model
      \begin{outline}
      \item Box of particles, each of which can have energy 0 or $\epsilon$
      \item Thermodynamic state defined by number of elements $N$, and number of
        quanta $q$, $U=q\epsilon$
      \item Degeneracy of given $N$ and $q$ given by binomial distribution:
        \begin{displaymath}
          \Omega=\frac{N!}{q!(N-q)!}
        \end{displaymath}
      \item Allow energy to flow between two such systems
        \begin{outline}
        \item Energy of a closed system is conserved (first law!)
        \item Degeneracy of total system is always $\geq$ degeneracy of the
          starting parts!
        \item Boltzmann's tombstone, $S = k_B \ln \Omega$
        \item Clausius: entropy of the universe seeks a maximum!  Second Law...
        \end{outline}
      \end{outline}
      \item Energy flow/thermal equilibrium between two large systems
        \begin{outline}
          \item Each subsystem has energy $U_i$ and degeneracy $\Omega_i(U_i)$
          \item Bring in thermal contact, $U=U_1+U_2$, $\Omega=\Omega_1(U_1)\Omega_2(U_2)$
          \item If systems are very large, one combination of $U_1$, $U_2$ and $\Omega$
            will be much more probably than all others
          \item What value of $U_1$ and $U_2=U-U_1$ maximizes $\Omega$?
        \begin{displaymath}
 \left ( \frac{\partial \ln \Omega_1}{\partial U_1} \right )_N = \left ( \frac{\partial \ln \Omega_2}{\partial U_2} \right )_N
        \end{displaymath}
        \begin{displaymath}
 \left ( \frac{\partial S_1}{\partial U_1} \right )_N = \left ( \frac{\partial S_2}{\partial U_2} \right )_N
        \end{displaymath}
      \item Thermal equilibrium is determined by equal {\bf temperature!}
        \begin{displaymath}
            \frac{1}{T}=\left ( \frac{\partial S}{\partial U} \right )_N
          \end{displaymath}
        \item When the temperatures of the two subsystems are equal, the
          entropy of the combined system is maximized!
        \item (Same arguments lead to requirement that equal pressures ($P_i$) and
          equal chemical potentials ($\mu_i$) maximize entropy when volumes or
          particles are exchanged)
        \end{outline}

      \item Two-state model in limit of large $N$
        \begin{outline}
        \item Large $N$ and Stirling's approximation
        \item Fundamental thermodynamic equation of two-state system:
        \begin{displaymath}
          S(U)=-k_B \left ( x \ln x + (1-x) \ln (1-x) \right ), \mathrm{where}\
          x = q/N = U/N\epsilon
        \end{displaymath}
      \item Temperature is derivative of entropy wrt energy yields          
          \begin{displaymath}
            U(T) = \frac{N\epsilon}{1+e^{\epsilon/k_BT}}
          \end{displaymath}
        \begin{outline}
          \item $T \rightarrow 0, U \rightarrow 0, S \rightarrow 0$, minimum disorder
          \item $T \rightarrow \infty, U \rightarrow N\epsilon/2, S \rightarrow
            k_B \ln 2$, maximum disorder
        \end{outline}
      \item Differentiate again to get heat capacity
      \end{outline}

    \item Canonical ($NVT$) ensemble
      \begin{outline}
      \item Previous is example of microcanonical (``$NVE$'') ensemble
      \item Direct evaluation of $S(U)$ is generally intractable, so seek simpler approach
      \item Imagine a system brought into thermal equilibrium with a much
        larger ``reservoir'' of constant $T$, such that the aggregate has a
        total energy $U$
      \item Degeneracy of a given system microstate $j$ with energy $U_j$
        is $\Omega_{res}(U-U_j)$
        \begin{eqnarray*}
          T = \frac{dU_{res}}{k_Bd\ln\Omega_{res}} \\
          \Omega_{res}(U-U_j) \propto e^{-U_j/k_B T}
        \end{eqnarray*}
      \item Probability for system to be in a microstate with energy $U_j$ given by Boltzmann
        distribution!
        \begin{displaymath}
          P(U_j) \propto e^{-U_j/k_B T} = e^{-U_j \beta}
        \end{displaymath}
      \item Partition function ``normalizes'' distribution, $Q(T) = \sum_j
        e^{-U_j \beta}$
      \item For system of identical (distinguishable) elements with energy states $\epsilon_i$,
        can factor probability to show
        \begin{eqnarray*}
          P(\epsilon_i) \propto e^{-\epsilon_i/k_B T} = e^{-\epsilon_i \beta},\
          \ \ \ \ \beta=1/k_BT
        \end{eqnarray*}
      \end{outline}


\item{Energy factoring}
  \begin{outline}
  \item{If system is large, how to determine it's energy states $U_j$?  There
      would be many, many of them!}
  \item{One simplification is if we can write energy as sum of energies of
      individual elements (atoms, molecules) of system:}
    \begin{align}
      U_j&=\epsilon_j(1)+\epsilon_j(2) + ... + \epsilon_j(N) \\
      Q(N,V,T) &= \sum_j e^{-U_j\beta} \\
      &=\sum_je^{-(\epsilon_j(1)+\epsilon_j(2) + ... + \epsilon_j(N))\beta}
    \end{align}
    \begin{outline}
    \item{{\em If} molecules/elements of system can be distinguished from each
        other (like atoms in a fixed lattice), expression can be factored:}
      \begin{align}
        Q(N,V,T)&=\left ( \sum_j e^{-\epsilon_j(1)\beta}\right )\cdots \left ( \sum_j
          e^{-\epsilon_j(N)\beta}\right ) \\
      &= q(1)\cdots q(N) \\
      \text{Assuming all the elements are the same:}\\
      &= q^N \\
     q&=\sum_j e^{-\epsilon_j \beta}: \mathrm{molecular\ partition\ function}
   \end{align}
  \item{{\em If not} distinguishable (like molecules in a liquid or gas, or
      electrons in a solid), problem is difficult, because identical
      arrangements of energy amongst elements should only be counted once.
      Approximate solution, good almost all the time:}
    \begin{equation}
      Q(N,V,T)=q^N/N!
    \end{equation}
  \item{Sidebar: ``Correct'' factoring depends on whether individual elements
      are fermions or bosons, leads to funny things like superconductivity and
      superfluidity.}
  \end{outline}
\end{outline}
% \item{Molecular partition function}
%   \begin{outline}
%   \item{Sum over energy states of single molecule/element of system}
%     \begin{equation}
%       q=\sum_j e^{-\epsilon_j \beta}: \mathrm{molecular\ partition\ function}
%     \end{equation}
%   \item{{\em This} can be evaluated for our QM energy models}
%   \end{outline}

    \item Two-state system again
      \begin{outline}
      \item Partition function, $q(T)=1+e^{-\epsilon\beta}$
      \item State probabilities
      \item{Internal energy $U(T)$}
        \begin{equation}
          U(T)=-N \left ( \frac{\partial \ln(1+e^{-\epsilon\beta})}{\partial\beta}
          \right)=\frac{N\epsilon e^{-\epsilon\beta}}{1+e^{-\epsilon\beta}}
        \end{equation}
     \item Heat capacity $C_v$
        \begin{outline}
        \item Minimum when change in states with $T$ is small
        \item Maximize when chagne in states with $T$ is large
        \end{outline}
      \item Helmholtz energy, $A= -\ln q/\beta$, decreasing function of $T$
      \item Entropy
      \end{outline}
    \item Distinguishable vs.\ indistinguishable particles
      \begin{outline}
      \item Distinguishable (e.g., in a lattice): $Q(N,V,T) = q(V,T)^N$
      \item Indistinguishable (e.g., a gas): $Q(N,V,T)\approx q(V,T)^N/N!$
      \end{outline}
      \item Thermodynamic functions in canonical ensemble
    \end{outline}

\begin{table}\small
  \begin{center}
    \caption{Equations of the Canoncial ($NVT$) Ensemble}
    \label{Canonical}
    \begin{tabular}[h]{lccc}
      \hline
$\beta=1/k_BT$ & {\bf Full Ensemble} & {\bf Distinguishable particles} & {\bf Indistinguishable
particles} \\
               &               & (e.g. atoms in a lattice) & (e.g. molecules in
               a fluid) \\
\hline
Single particle & & & \\partition function& & $\displaystyle q(V,T) = \sum_i
e^{-\epsilon_i\beta} $& $\displaystyle q(V,T) = \sum_i e^{-\epsilon_i\beta} $ \\
Full partition & & & \\function & $\displaystyle Q(N,V,T) = \sum_j e^{-U_j\beta} $ &
$\displaystyle Q = q(V,T)^N $ & $\displaystyle Q = q(V,T)^N/N! $ \\
Log partition &  $\ln Q$ & $N\log q$ & $ N\ln q - \ln N! $\\
function & & & $\approx N(\ln Q - \ln N +1)$ \\ & & & \\
Helmholtz energy & $\displaystyle -\frac{\ln Q}{\beta}$ & $\displaystyle
-\frac{N\ln q}{\beta}$ & $\displaystyle -\frac{N}{\beta}\left (\ln\frac{q}{N} +
  1 \right ) $ \\
($A=U-TS$) & & & \\ & & &  \\
Internal energy ($U$)& $\displaystyle -\left (\frac{\partial\ln
    Q}{\partial\beta}\right )_{NV}$ & $\displaystyle -N\left (\frac{\partial\ln
    q}{\partial\beta}\right )_{V}$ &  $\displaystyle -N\left (\frac{\partial\ln
    q}{\partial\beta}\right )_{V}$ \\ & & & \\
Pressure ($P$) & $\displaystyle -\left (\frac{\partial\ln
    Q}{\partial V}\right )_{N\beta}$ & $\displaystyle -N\left (\frac{\partial\ln
    q}{\partial V}\right )_{\beta}$ &  $\displaystyle -N\left (\frac{\partial\ln
    q}{\partial V}\right )_{\beta}$ \\ & & & \\

Entropy ($S/k_B$) & $ \beta U + \ln Q$ & $\beta U + N \ln q$ & $\beta U +
N\left ( \ln(q/N) + 1\right )$ \\ & & & \\
Chemical potential ($\mu$) & $\displaystyle -\frac{1}{\beta}\left ( \frac{\partial \ln
    Q}{\partial N}\right )_{VT} $& $\displaystyle -\frac{\ln q}{\beta}$ & $\displaystyle
-\frac{\ln (q/N)}{\beta}$ \\ & & & \\
\hline
    \end{tabular}
{\bf NOTE!} All energies are referenced to their values at 0~K.  Enthalpy $H=U+PV$, Gibb's
Energy $G=A+PV$.
  \end{center}
\end{table}

  \item{\bf Lecture 16: Molecular partition functions}
    \begin{outline}
    \item Ideal gas of molecules
      \begin{displaymath}
        Q_{ig}(N,V,T) = \frac{(q_\mathrm{trans}q_\mathrm{rot}q_\mathrm{vib})^N}{N!}
      \end{displaymath}

      \item Particle-in-a-box (translational states of a gas)
        \begin{outline}
          \item Energy states $\epsilon_n=n^2\epsilon_0, n=1,2, \ldots$,
            $\epsilon_0$ tiny for macroscopic $V$
          \item $\Theta_\mathrm{trans} = \epsilon_0/k_B$ translational temperature
          \item $\Theta_\mathrm{trans} << T \rightarrow$ {\em many} states contribute
            to $q_\mathrm{trans}\rightarrow$ integral approximation
            \begin{eqnarray*}
              q_\mathrm{trans,1D} = \int_0^\infty e^{-x^2\beta\epsilon_0}dx =
              L/\Lambda \\
              \Lambda = \left ( \frac{h^2\beta}{2\pi m} \right )^{1/2}\
              \mathrm{thermal\ wavelength} \\
              q_\mathrm{trans,3D} = V/\Lambda^3
            \end{eqnarray*}
          \item Internal energy
          \item Heat capacity
          \item Equation of state (!)
          \item Entropy: Sackur-Tetrode equation
        \end{outline}
      \item Rigid rotor (rotational states of a gas)
        \begin{outline}
        \item energy states and degeneracies
        \item $\Theta_\mathrm{rot} = \hbar^2/2 I k_B$
        \item ``High'' T $q_\mathrm{rot}(T) \approx \sigma \Theta_\mathrm{rot}/T$
        \end{outline}
      \item Harmonic oscillator (vibrational states of a gas)
        \begin{outline}
          \item $\Theta_\mathrm{vib}=h\nu/k_B$
        \end{outline}

      \item Electronic partition functions $\rightarrow$ spin multiplicity
      \item Non-ideality
        \begin{outline}
          \item Real molecules interact through vdW interactions
          \item Particle-in-a-box model breaks down, have to work harder but
            can still get at same ideas
          \item See Hill, {\em J. Chem. Ed.} {\bf 1948}, {\em 25}, p. 347, http://dx.doi.org/10.1021/ed025p347
        \end{outline}
      \end{outline}

\begin{table} 
\begin{center}
    \caption{\large{Statistical Thermodynamics of an Ideal Gas}}
   \begin{description}
    \item[\underline{Translational DOFs}] {3-D particle in a box model}

$\displaystyle \theta_\mathrm{trans}= \frac{\pi^2\hbar^2}{2 m
  L^2 k_B}$, 
$\displaystyle \Lambda=h\left( \frac{\beta}{2\pi m}\right )^{1/2}$

For $ T >> \Theta_\mathrm{trans}$, $\Lambda << L$, $\displaystyle
q_\mathrm{trans}=V/\Lambda^3$ (essentially always true)

\begin{tabular}{ccc}
$\displaystyle U_\mathrm{trans}=\frac{3}{2}RT$ & $\displaystyle C_\mathrm{v,trans} =
\frac{3}{2}R $ & $\displaystyle S^\circ_\mathrm{trans}=R \ln \left (
  \frac{e^{5/2}V^\circ}{N^\circ \Lambda^3}\right ) = R \ln \left (
  \frac{e^{5/2}k_BT}{P^\circ \Lambda^3}\right ) $ \\
\end{tabular}

  \item[\underline{Rotational DOFs}] {Rigid rotor model}
\begin{description}
\item[Linear molecule]{}
$\theta_\mathrm{rot} =hcB/k_B$

\begin{equation*}
q_\mathrm{rot}=\frac{1}{\sigma}\sum_{l=0}^\infty (2l+1)e^{-l(l+1)\theta_\mathrm{rot}/T},  
\approx \frac{1}{\sigma}\frac{T}{\theta_\mathrm{rot}},\ \ T>>\theta_\mathrm{rot}\ \ \ \sigma = \left \{
        \begin{array}{rl}
          1, & \text{unsymmetric} \\
          2, & \text{symmetric}
        \end{array} \right . 
\end{equation*}
\begin{tabular}{ccc}
$\displaystyle U_\mathrm{rot}=RT$ & $\displaystyle C_\mathrm{v,rot} =
R $ & $\displaystyle S^\circ_\mathrm{rot}=R (1-\ln(\sigma\theta_\mathrm{rot}/T)) $ \\
\end{tabular}

\item[Non-linear molecule]{} $\theta_{\mathrm{rot},\alpha}=hcB_\alpha/k_B$
\begin{equation*}
q_\mathrm{rot} 
\approx \frac{1}{\sigma}\left ( \frac{\pi
    T^3}{\theta_{\mathrm{rot},\alpha}\theta_{\mathrm{rot},\beta}\theta_{\mathrm{rot},\gamma}}
  \right )^{1/2},\ \ T>>\theta_{\mathrm{rot},\alpha,\beta,\gamma}\ \ \ \sigma =
  \text{rotational symmetry number}
\end{equation*}
\begin{tabular}{ccc}
$\displaystyle U_\mathrm{rot}=\frac{3}{2}RT$ & $\displaystyle C_\mathrm{v,rot} = \frac{3}{2}
R $ & $\displaystyle S^\circ_\mathrm{rot}=\frac{R}{2}
\left ( 3-\ln\frac{\sigma\theta_{\mathrm{rot},\alpha}\theta_{\mathrm{rot},\beta}\theta_{\mathrm{rot},\gamma}}{\pi
  T^3} \right ) $ \\
\end{tabular}

\end{description}

\item[\underline{Vibrational DOFs}] {Harmonic oscillator model}
\begin{description}
\item[Single harmonic mode] {$\theta_\mathrm{vib}=h\nu/k_B $}
  \begin{equation*}
    q_\mathrm{vib}=\frac{1}{1-e^{-\theta_\mathrm{vib}/T}} \approx
      \frac{T}{\theta_\mathrm{vib}}, \ \ \ T>>\theta_\mathrm{vib}
  \end{equation*}

\begin{tabular}{ccc}
$ U_\mathrm{vib}= $ & $  C_\mathrm{v,vib} = $ & $S^\circ_{\mathrm{vib},i}=$ \\
$\displaystyle
R\frac{\theta_\mathrm{vib}}{e^{\theta_\mathrm{vib}/T}-1}$ &
$\displaystyle R\left (
  \frac{\theta_\mathrm{vib}}{T}\frac{e^{\theta_\mathrm{vib}/2T}}{e^{\theta_\mathrm{vib}/T}-1}
\right )^2 $ & $\displaystyle R \left ( \frac{\theta_\mathrm{vib}/T}{e^{\theta_\mathrm{vib}/T}-1}
-\ln(1-e^{-\theta_\mathrm{vib}/T})\right ) $ \\
\end{tabular}

\item[Multiple harmonic modes] {$\theta_{\mathrm{vib},i}=h\nu_i/k_B $}

  \begin{equation*}
    q_\mathrm{vib}=\prod_i\frac{1}{1-e^{-\theta_{\mathrm{vib},i}/T}} 
  \end{equation*}

\begin{tabular}{ccc}
$ U_\mathrm{vib}= $ & $  C_\mathrm{v,vib} = $ & $S^\circ_{\mathrm{vib},i}=$ \\
$\displaystyle
R\sum_i\frac{\theta_{\mathrm{vib},i}}{e^{\theta_{\mathrm{vib},i}/T}-1}$ &
$\displaystyle R \sum_i \left (
  \frac{\theta_{\mathrm{vib},i}}{T}\frac{e^{\theta_{\mathrm{vib},i}/2T}}{e^{\theta_{\mathrm{vib},i}/T}-1}
\right )^2 $ & $\displaystyle R \left ( \frac{\theta_{\mathrm{vib},i}/T}{e^{\theta_{\mathrm{vib},i}/T}-1}
-\ln(1-e^{-\theta_{\mathrm{vib},i}/T})\right ) $ \\
\end{tabular}

\end{description}
\item[\underline{Electronic DOFs}] {}
$q_\mathrm{elec} = \text{spin multiplicity}$


\end{description}
\end{center}
\end{table}


    \item {\bf Lecture 17: Chemical reactions and equilibrium}
      \begin{outline}
      \item Standard states
        \begin{outline}
          \item Translational partition function depends on concentration $N/V$
          \item ``Standard state'' corresponds to some standard choice for $N/V$, $c^\circ$
          \item For ideal gas, related to pressure by $P^\circ = c^\circ k_B T$
        \end{outline}
      \item Chemical reaction $A \rightarrow B$
      \item Reaction entropy $\Delta S^\circ (T) =  S^\circ_\mathrm{B}(T)-S^\circ_\mathrm{A}(T)$
        \item Reaction energy $\Delta U^\circ (T) =
          U^\circ_\mathrm{B}(T)-U^\circ_\mathrm{A}(T)+\Delta E(0)$
        \item Equilibrium condition---equate chemical potentials, $\mu_A(N,V,T) = \mu_B(N,V,T)$
        \item Equilibrium constant---evaluate from partition functions directly
          or indirectly from thermodynamic potentials
\item Le'Chatlier's principle
  \begin{outline}
    \item Response to temperature: Boltzmann distribution favors higher energy
      things as $T$ increases
    \item Response to volume chance: particle-in-a-box states increasingly favor
      side with more molecules as volume increases 
  \end{outline}
\end{outline}
\item {\bf Lecture 18: Chemical kinetics}
  \begin{outline}
  \item Kinetics and reaction rates
    \begin{outline}
      \item Rate: number per unit time per unit something
    \end{outline}

  \item Empirical chemical kinetics
    \begin{outline}
    \item Rate laws, rate orders, and rate constants
    \item Arrhenius expression, $k=A e^{-E_a/k_BT}$
    \end{outline}
  \item Reaction mechanisms
  \item Elementary steps and molecularity
  \item Collision theory---overpredicts rates
  \item Transition state theory (TST)
    \begin{outline}
    \item Existence of reaction coordinate (PES)
    \item Existence of dividing surface
    \item Equilibrium between reactants and ``transition state''
    \item Harmonic approximation for transition state
    \end{outline}
  \item Locating transition states computationally
  \item Thermodynamic connection 
  \item (Skipped) Diffusion-controlled reactions 
    \begin{outline}
      \item Intermediate complex
      \item Steady-state approximation
      \item Diffusion-controlled limit ($k_D = 4\pi (r_A + r_B) D_{AB}$)
      \item Reaction-controlled limit ($k_{app}=(k_D/k_{-D})k_r$)
    \end{outline}

  \end{outline}

\begin{table} 
\begin{center}
    \caption{\large{Equilibrium and Rate Constants}}
   \begin{description}
   \item[Equilibrium Constants] $a~\text{A} + b~\text{B} \rightleftharpoons c~\text{C} + d~\text{D} $
     \begin{eqnarray*}
       K_{eq}(T) &=& e^{\Delta S^\circ(T,V)/k_B}e^{-\Delta H^\circ(T,V)/k_BT}
       \\ \\ 
            K_c(T) &=&
          \left(\frac{1}{c^\circ}\right)^{\nu_c+\nu_d-\nu_a-\nu_b}\frac{(q_c/V)^{\nu_c}(q_d/V)^{\nu_d}}{(q_a/V)^{\nu_a}(q_b/V)^{\nu_b}}e^{-\Delta
            E(0)\beta}\\ \\
            K_p(T) &=&
          \left(\frac{k_BT}{P^\circ}\right)^{\nu_c+\nu_d-\nu_a-\nu_b}\frac{(q_c/V)^{\nu_c}(q_d/V)^{\nu_d}}{(q_a/V)^{\nu_a}(q_b/V)^{\nu_b}}e^{-\Delta
            E(0)\beta}
\end{eqnarray*}
\item[Unimolecular Reaction] $\text[A] \rightleftharpoons [\text{A} ]^\ddagger
  \rightarrow C$
      \begin{displaymath}
        k(T)=\nu^\ddagger \bar K^\ddagger=\frac{k_B T}{h} \frac{\bar{q}_\ddagger(T)/V}{q_A(T)/V}
          e^{-\Delta E^\ddagger(0)\beta}               
      \end{displaymath}
\begin{center}
      \begin{tabular}{cc}
      $ \displaystyle E_a =\Delta H^{\circ\ddagger}+k_B T $
      & $ \displaystyle A = e^1\frac{k_B T}{h} e^{\Delta S^{\circ\ddagger}} $
      \end{tabular}
\end{center}
\item[Bimolecular Reaction] $
        \mathrm{A} + \mathrm{B} \rightleftharpoons [ \mathrm{AB}]^\ddagger
        \rightarrow \text{C}$
      \begin{displaymath}
        k(T)=\nu^\ddagger \bar K^\ddagger=\frac{k_B T}{h} \frac{q_\ddagger(T)/V}{(q_A(T)/V)(q_B(T)/V)}\left
          (\frac{1}{c^\circ}\right )^{-1}
        e^{-\Delta E^\ddagger(0)\beta}               
      \end{displaymath}
      \begin{center}
        \begin{tabular}{cc}
        $ \displaystyle E_a  =\Delta H^{\circ\ddagger}+2 k_B T $ & $ \displaystyle
        A  = e^2\frac{k_B T}{h} e^{\Delta S^{\circ\ddagger}} $
      \end{tabular}
      \end{center}
   \end{description}
 \end{center}
 \end{table}

\item {\bf Lecture 19: Conclusion}
  \begin{outline}
    \item Do you think about the burning lighter any differently now?  
  \end{outline}

\end{outline}
\end{document}
\message{ !name(Outline.tex) !offset(-1606) }
